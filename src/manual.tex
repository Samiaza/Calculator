\documentclass{article}
\usepackage{geometry}
\usepackage{indentfirst}
\geometry{verbose,a4paper,tmargin=2cm,bmargin=2cm,lmargin=2.5cm,rmargin=1.5cm}
\title{Quick guide to the Calculator}

\author{Aleksei Istomin aka Samiaza aka Gemma Emery}

\date{April 2023}

\begin{document}

\maketitle
\pagebreak

% \tableofcontents
% \pagebreak


\section{Introduction}
The Calculator is an extended version of the usual calculator, which can be found in the standard applications of each operating system. The Core was implemented in the C++ programming language using structured programming. The GUI was implemented using the Qt library. In addition to basic arithmetic operations such as add/subtract and multiply/divide, this calculator was supplemented with the ability to calculate arithmetic expressions by following the order, as well as some mathematical functions (sine, cosine, logarithm, etc.).

Besides calculating expressions, it also supports the use of the X variable and the graphing of the corresponding function.

Other improvements include a credit and deposit calculators.

% \pagebreak

\section{Arithmetic operations and mathematical functions}

The following arithmetic operations and mathematical functions are supported:
  \begin{itemize}
    \item Arithmetic operators
    \begin{itemize}
      \item Brackets -- (a + b)
      \item Addition -- a + b
      \item Subtraction -- a - b
      \item Multiplication -- a * b
      \item Division -- a / b
      \item Power -- a \^\ b
      \item Modulus -- a \% b
      \item Unary plus -- +a
      \item Unary minus -- -a
    \end{itemize}
    \item Functions
    \begin{itemize}
      \item Cosine calculation -- cos(x)
      \item Sine calculation -- sin(x)
      \item Tangent calculation -- tan(x)
      \item Arc cosine calculation -- acos(x)
      \item Arc sine calculation -- asin(x)
      \item Arc tangent calculation -- atan(x)
      \item Square root calculation -- sqrt(x)
      \item Natural logarithm calculation -- ln(x)
      \item Common logarithm calculation -- log(x)
    \end{itemize}
  \end{itemize}

% \pagebreak

\section{Plotting}

There is implemented plotting a graph of a function given by an expression in infix notation with the variable X (with coordinate axes, mark of the used scale and an adaptive grid).
\pagebreak
\section{Bonus. Credit calculator}
The Calculator provides a special mode "credit calculator":
  \begin{itemize}
    \item Input: total credit amount, term, interest rate, type (annuity, differentiated)
    \item Output: monthly payment, overpayment on credit, total payment
  \end{itemize}

\section{Bonus. Deposit calculator}
The Calculator provides a special mode "deposit profitability calculator":
  \begin{itemize}
    \item Input: deposit amount, deposit term, interest rate, tax rate, periodicity of payments, capitalization of interest, replenishments list, partial withdrawals list
    \item Output: accrued interest, tax amount, deposit amount by the end of the term
  \end{itemize}
  
\end{document}

